\problemname{A1 Paper}

\illustration{.45}{A-size_paper}{Illustration of the A series paper sizes by \href{http://commons.wikimedia.org/wiki/File:A_size_illustration2.svg}{Bromskloss}}

Björn likes the square root of two, $\sqrt{2} = 1.41421356\dots$ very much. He likes it so much that
he has decided to write down the first $10\,000$ digits of it on a single paper. He started doing
this on an A4 paper, but ran out of space after writing down only $1250$ digits. Being pretty good at
math, he quickly figured out that he needs an A1 paper to fit all the digits. Björn doesn't have an A1
paper, but he has smaller papers which he can tape together.

Taping two A2 papers together along their long side turns them into an A1 paper, two A3 papers give
an A2 paper, and so on. Given the number of papers of different sizes that Björn has, can you figure
out how much tape he needs to make an A1 paper? Assume that the length of tape needed to join
together two sheets of papers is equal to their long side. An A2 paper is $2^{-5/4}$~meters by
$2^{-3/4}$~meters and each consecutive paper size (A3, A4, \dots) have the same shape but half the
area of the previous one.

\section*{Input}
The first line of input contains a single integer $2\leq n \leq 30$, the A-size of the smallest
papers Björn has. The second line contains $n-1$ integers giving the number of sheets he has of each
paper size starting with A2 and ending with A$n$. Björn doesn't have more than $10^9$ sheets of any
paper size.

\section*{Output}
If Björn has enough paper to make an A1 paper, output a single floating point number, the smallest
total length of tape needed in meters. Otherwise output ``\texttt{impossible}''. The output number should have an absolute error of at most $10^{-5}$.
